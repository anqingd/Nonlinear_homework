\begin{itemize}
  \item predator-prey ($x$ predator, $y$ prey)\\
  $c$ is positive and $e$ is negative. Large population of $y$ benefits $x$ since there are sufficient food, which implies that the population of $x$ shall grow faster when $y$ increases. Therefore $c$ must be positive to show the positive impact $y$ has on $\dot{x}$. However, the more $x$, the worse situation for $y$. Increasing of $x$ makes it harder for $y$ to survive. Therefore, $e$ is negative.
  \item prey-predator ($x$ prey, $y$ predator)\\
  $c$ is negative and $e$ is positive. The elaboration is the same as previous case except that $x$ and $y$ have inverted roles here.
  \item competitive ($x$ and $y$ inhibit each other)\\
  $c$ and $e$ are both negative. Since $x$ and $y$ compete for some common source, increase of one species means higher inhibitive effect on the other, which indicates that $c$ and $e$ shall both be negative.
  \item symbiotic ($x$ and $y$ benefit each other)\\
  $c$ and $e$ are both positive. The existence of one species benefits the other, which means the population of one species shall grow faster if the other increases. Therefore both $c$ and $e$ shall be positive to reflect the favourable impact they have on each other.
\end{itemize}