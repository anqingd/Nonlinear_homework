\begin{enumerate}

\item The phase portrait is shown as figure ~\ref{fig:pplane5}.
\item When 
$a=e=1$, $b=f=0$, and $c=d=-1$,
  we can obtain the differential equations as \eqref{eqn:PPS1} and \eqref{eqn:PPS2}.
\begin{subequations} \label{eqn:PPS}
\begin{align}
    \dot{x} &= x(1-y)  \label{eqn:PPS1}\\ 
    \dot{y} &= y(-1+x) \label{eqn:PPS2}
\end{align}
\end{subequations}
By calculating $(x^\star,y^\star)$, the system has two equilibrium points $(0,0)$ and $(1,1)$. The Jocabian matrix can be derived in a similar way. The eigenvalues are $\lambda_{1,2} = \pm 1$ and $\lambda_{1,2} = \pm j$ respectively. It can be derived that the equilibrium $(0,0)$ is a saddle point. For the equilibrium $(1,1)$, since it is a non-hyperbolic point, the stability needs to be checked with other methods. From figure ~\ref{fig:pplane5} we can infer that this equilibrium is a stable point. 
\item The physical species model in this case could be $y$ preying on $x$ and $x$ feeding grass which is infinite. When $x$ is comparatively big, $y$ would increase to control the population of $x$; when $y$ is comparatively big, $y$ would decrease due to the intraspecies competition and lack of food, which would promote the breeding of $x$. Therefore, the population would stay stable under this natural negative feedback effect.  
\end{enumerate}