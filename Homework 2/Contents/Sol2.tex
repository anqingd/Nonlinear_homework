When $a=3$, $b=f=-1$, and $d=2$ we can obtain the differential equations as  \eqref{eqn:PPSystem11} and \eqref{eqn:PPSystem12}.
\begin{subequations}\label{eqn:PPSystem1}
\begin{align}
    \dot{x} = x(3-x+cy) \label{eqn:PPSystem11} \\
    \dot{y} = y(2+ex-y) \label{eqn:PPSystem12}
\end{align}
\end{subequations}
The Jacobian matrix is defined as $A = \left. \frac{\partial\textbf{f}}{\partial \textbf{x}}(\textbf{x}) \right|_{\textbf{x}=\textbf{x}_0}$, which in those cases is
\begin{equation*}
    \frac{\partial\textbf{f}}{\partial \textbf{x}}(\textbf{x}) =
    \left[\begin{array}{cc}
    3-2x+cy & cx \\
    ey & 2+ex-2y
    \end{array}\right],
\end{equation*}
The eigenvalues of the Jacobian matrix are given by the characteristic equation
\begin{equation*}
    \det(\lambda I - A) = 0.
\end{equation*}
Note that the phase portrait is only drawn at $x\geq0$ and $y\geq0$ for each case, since the physical context is meaningful only when the population is non-negative.
\subsection*{Case 1: $(c,e)=(-2,-1)$}
%(-2,-1)
\begin{enumerate}
\item The phase portrait is shown as figure ~\ref{fig:pplane1}.
\item When $(c,e)=(-2,-1)$, the Jacobian matrix can be written as
\begin{equation*}
    \frac{\partial\textbf{f}}{\partial \textbf{x}}(\textbf{x}) =
    \left[\begin{array}{cc}
    3-2x-2y & -2x \\
    -y & 2-x-2y
    \end{array}\right].
\end{equation*}
There are four equilibrium points $(x^\star,y^\star)$ in total. They are $(0,0),(0,2),(3,0)$ and $(1,1)$ respectively. The Jocabian matrices are

\begin{equation*}
    A_{(0,0)} =
    \left[\begin{array}{cc}
    3 & 0 \\
    0 & 2
    \end{array}\right], \;\;\quad\, A_{(0,2)} =
    \left[\begin{array}{cc}
    -1 & 0 \\
    -2 & -2
    \end{array}\right],   
    \end{equation*}
    \begin{equation*}
    A_{(3,0)} =
    \left[\begin{array}{cc}
    -3 & -6 \\
    0 & -1
    \end{array}\right], \;A_{(1,1)} =
    \left[\begin{array}{cc}
    -1 & -2 \\
    -1 & -1
    \end{array}\right].
\end{equation*}

The characteristic equation of the system linearized around $(0,0)$ is $$\lambda^2 -5 \lambda + 6 = 0,$$ which gives the eigenvalues $\lambda_{1} = 3$ and $\lambda_{2} = 2$. The equilibrium $(0,0)$ is an unstable node. 

The characteristic equation of the system linearized around $(0,2)$ is $$\lambda^2 +3 \lambda + 2 = 0,$$ which gives the eigenvalues $\lambda_1 = -1$ and $\lambda_2 = -2$. The equilibrium $(0,2)$ is a stable node.
 
The characteristic equation of the system linearized around $(3,0)$ is
$$\lambda^2 +4 \lambda + 3 = 0,$$ which gives the eigenvalues $\lambda_1 = -1$ and $\lambda_2 = -3$. The equilibrium $(3,0)$ is a stable node.

The characteristic equation of the system linearized around $(1,1)$ is $$\lambda^2 +2 \lambda - 1 = 0,$$ which gives the eigenvalues $\lambda_1 = \sqrt{2} -1 \approx 0.414$ and $\lambda_2 = -1 -\sqrt{2} \approx -2.414$. The equilibrium $(1,1)$ is a saddle point.

\item In this case, the relationship between $x$ and $y$ is competitive. This implies that one species would always tend to become dominate and the other would become extinct based on the initial ratio. There is one point where two species could co-exist but it$'$s not stable since the equilibrium could be broken easily as long as small number disturbance happens.  
\end{enumerate}

\subsection*{Case 2: $(c,e)=(-2,1)$}
%(-2,1)
\begin{enumerate}
\item The phase portrait is shown as figure ~\ref{fig:pplane2}.
\item When $(c,e)=(-2,1)$, there are three equilibrium points $(x^\star,y^\star)$ in total. They are $(0,0)$, $(0,2)$ and $(3,0)$ respectively. The Jocabian matrix can be derived similarly. The corresponding eigenvalues for each equilibrium point are $\lambda_{1} = 3$, $\lambda_{2} = 2$; $\lambda_1 = -1$, $\lambda_2 = -2$; $\lambda_1 = -3$, $\lambda_2 = 5$. Therefore the equilibrium points are unstable node, stable node and saddle point respectively.
\item In this case, the relationship between the two species is $y$ preying on $x$. This implies that $y$ would increase faster if the relative population of $x$ is bigger and similarly the amount of $x$ would decrease faster if if the relative population of $y$ is bigger.
\end{enumerate}

\subsection*{Case 3: $(c,e)=(2,-1)$}
%(2,-1)
\begin{enumerate}
\item The phase portrait is shown as figure ~\ref{fig:pplane3}.
\item When $(c,e)=(2,-1)$, there are three equilibrium points $(x^\star,y^\star)$ in total. They are $(0,0)$, $(0,2)$ and $(3,0)$ respectively. The Jocabian matrix can be derived similarly. The corresponding eigenvalues for each equilibrium point are $\lambda_{1} = 3$, $\lambda_{2} = 2$; $\lambda_1 = -2$, $\lambda_2 = 7$; $\lambda_1 = -3$, $\lambda_2 = -1$. Therefore the equilibrium points are unstable node, saddle node and stable point respectively. 
\item In this case, the relationship between the two species is $x$ preying on $y$. This implies that $x$ tends to increase faster if the relative amount of $y$ is bigger and similarly the amount of $y$ would decrease faster if the relative amount of $x$ is bigger.   
\end{enumerate}

\subsection*{Case 4: $(c,e)=(2,1)$}
%(2,1)
\begin{enumerate}
\item The phase portrait is shown as figure ~\ref{fig:pplane4}. 
\item When $(c,e)=(2,-1)$, there are three equilibrium points $(x^\star,y^\star)$ in total. They are $(0,0)$, $(0,2)$ and $(3,0)$ respectively. The Jocabian matrix can be derived similarly. The corresponding eigenvalues for each equilibrium point are $\lambda_{1} = 3$, $\lambda_{2} = 2$; $\lambda_1 = -2$, $\lambda_2 = 7$; $\lambda_1 = -3$, $\lambda_2 = 5$. Therefore the equilibrium points are unstable node, saddle node and saddle node respectively. 
\item In this case, the relationship between the two species is symbiotic. This implies that one species would increase faster as the other one increases. 
\end{enumerate}