\documentclass[a4paper,twocolumn]{article} % Document type

\ifx\pdfoutput\undefined
    %Use old Latex if PDFLatex does not work
   \usepackage[dvips]{graphicx}% To get graphics working
   \DeclareGraphicsExtensions{.eps} % Encapsulated PostScript
 \else
    %Use PDFLatex
   \usepackage[pdftex]{graphicx}% To get graphics working
   \DeclareGraphicsExtensions{.pdf,.jpg,.png,.mps} % Portable Document Format, Joint Photographic Experts Group, Portable Network Graphics, MetaPost
   \pdfcompresslevel=9
\fi
\usepackage{amsmath}
\usepackage{cleveref}
\usepackage{amsmath,amssymb}   % Contains mathematical symbols
\usepackage[ansinew]{inputenc} % Input encoding, identical to Windows 1252
\usepackage[english]{babel}    % Language
%\usepackage[round,authoryear]{natbib}  %Nice author (year) citations
\usepackage[square,numbers]{natbib}     %Nice numbered citations
%\bibliographystyle{unsrtnat}           %Unsorted bibliography
\bibliographystyle{plainnat}            %Sorted bibliography

\addtolength{\topmargin}{-30mm}% Removes 30mm from the top margin
\addtolength{\textheight}{30mm}% Adds it to the text height


\begin{document}

\section{Problem 2 - Phase portrait and equilibrium interpretation for different cases}
\label{sec:prob2}
The given population model will be analyzed for different signs combination of $c$ and $e$, which would lead to four cases in total. Phase portrait would be obtained by using pplane for each case. And the equilibrium point(s) in each case should be identified analytically. The results should be interpreted under the physical context of two species. 


\section{Solution 2 - Phase portrait and equilibrium interpretation for different cases}
\label{sec:solu2}
When 
$a=3,$ $b=f=-1,$ and $d=2,$
  we can obtain the differential equations as \cref{eqn:PPSystem1,eqn:PPSystem2}\par 
\begin{subequations}\label{eqn:PPSystem}
\begin{align}
    \dot{x} = x(3-x+cy) \label{eqn:PPSystem1} \\
    \dot{y} = y(2+ex-y) \label{eqn:PPSystem2}
\end{align}
\end{subequations}
The Jacobian matrix is defined as $A = \left. \frac{\partial\textbf{f}}{\partial \textbf{x}}(\textbf{x}) \right|_{\textbf{x}=\textbf{x}_0}$. The Jacobian matrix is calculated to be
\begin{equation*}
    \frac{\partial\textbf{f}}{\partial \textbf{x}}(\textbf{x}) =
    \left[\begin{array}{cc}
    3-2x+cy & cx \\
    ey & 2+ex-2y
    \end{array}\right],
\end{equation*}
The eigenvalues of the Jacobian matrix are given by the characteristic equation
\begin{equation*}
    \det(\lambda I - A) = 0.
\end{equation*}
\subsection{(c,e)=(-2,-1)}
%(-2,-1)
When $(c,e)=(-2,-1)$, the Jacobian matrix can be written as
\begin{equation*}
    \frac{\partial\textbf{f}}{\partial \textbf{x}}(\textbf{x}) =
    \left[\begin{array}{cc}
    3-2x-2y & -2x \\
    -y & 2-x-2y
    \end{array}\right],
\end{equation*}
There are four equilibrium points $(x^*,y^*)$ in total. They are $(0,0),(0,2),(3,0)$and$(1,1)$ respectively. The Jocabian matrix in this case is
\begin{equation*}
    A_{(0,0)} =
    \left[\begin{array}{cc}
    3 & 0 \\
    0 & 2
    \end{array}\right], \; A_{(0,2)} =
    \left[\begin{array}{cc}
    -1 & 0 \\
    -2 & -2
    \end{array}\right],   
    \end{equation*}
    \begin{equation*}
    A_{(3,0)} =
    \left[\begin{array}{cc}
    -3 & -6 \\
    0 & -1
    \end{array}\right], \;A_{(1,1)} =
    \left[\begin{array}{cc}
    -1 & -2 \\
    -1 & -1
    \end{array}\right].
\end{equation*}

The characteristic equation of the system linearized around \mbox{(0,0)} is $\lambda^2 -5 \lambda + 6 = 0,$ which gives the eigenvalues $\lambda_{1} = 3$ and $\lambda_{2} = 2$. The equilibrium $(0,0)$ is an unstable node. 
The characteristic equation of the system linearized around \mbox{(0,2)} is $\lambda^2 +3 \lambda + 2 = 0,$ which gives the eigenvalues $\lambda_1 = -1$ and $\lambda_2 = -2$. The equilibrium $(0,2)$ is a stable node. 
The characteristic equation of the system linearized around \mbox{(3,0)} is
$\lambda^2 +4 \lambda + 3 = 0,$ which gives the eigenvalues $\lambda_1 = -1$ and $\lambda_2 = -3$. The equilibrium $(3,0)$ is a stable node.
The characteristic equation of the system linearized around \mbox{(1,1)} is $\lambda^2 +2 \lambda - 1 = 0,$ which gives the eigenvalues $\lambda_1 = \sqrt{2} -1 \approx 0.414$ and $\lambda_2 = -1 -\sqrt{2} \approx -2.414$. The equilibrium $(1,1)$ is a saddle point.

In this case, the relationship between $x$ and $y$ is competitive. This implies that one species would always tend to become dominate and the other would become extinct based on the initial ratio. There is one point where two species could co-exist but it$'$s not stable since the equilibrium could be broken easily as long as small number disturbance happens.  

\subsection{(c,e)=(-2,1)}
%(-2,1)
When $(c,e)=(-2,1)$, the Jacobian matrix can be written as
\begin{equation*}
    \frac{\partial\textbf{f}}{\partial \textbf{x}}(\textbf{x}) =
    \left[\begin{array}{cc}
    3-2x-2y & -2x \\
    y & 2+x-2y
    \end{array}\right],
\end{equation*}
There are three equilibrium points $(x^*,y^*)$ in total. They are $(0,0),(0,2)$ and $(3,0)$ respectively. The Jocabian matrix in this case is
\begin{equation*}
    A_{(0,0)} =
    \left[\begin{array}{cc}
    3 & 0 \\
    0 & 2
    \end{array}\right], \; A_{(0,2)} =
    \left[\begin{array}{cc}
    -1 & 0 \\
    2 & -2
    \end{array}\right],   
    \end{equation*}
    \begin{equation*}
    A_{(3,0)} =
    \left[\begin{array}{cc}
    -3 & -6 \\
    0 & 5
    \end{array}\right].
\end{equation*}

The characteristic equation of the system linearized around \mbox{(0,0)} is
$\lambda^2 -5 \lambda + 6 = 0,$ which gives the eigenvalues $\lambda_{1} = 3$ and $\lambda_{2} = 2$. The equilibrium $(0,0)$ is an unstable node. 
The characteristic equation of the system linearized around \mbox{(0,2)} is $\lambda^2 +3 \lambda + 2 = 0,$ which gives the eigenvalues $\lambda_1 = -1$ and $\lambda_2 = -2$. The equilibrium $(0,2)$ is a stable node. 
The characteristic equation of the system linearized around \mbox{(3,0)} is $\lambda^2 -2 \lambda - 15 = 0,$ which gives the eigenvalues $\lambda_1 = -3$ and $\lambda_2 = 5$. The equilibrium $(3,0)$ is a saddle point.

In this case, the relationship between the two species is $y$ preying on $x$. This implies that $y$ would increase if the relative amount of $x$ is huge and vice versa.

\subsection{(c,e)=(2,-1)}
%(2,-1)
When $(c,e)=(2,-1)$, the Jacobian matrix can be written as
\begin{equation*}
    \frac{\partial\textbf{f}}{\partial \textbf{x}}(\textbf{x}) =
    \left[\begin{array}{cc}
    3-2x+2y & 2x \\
    -y & 2-x-2y
    \end{array}\right],
\end{equation*}
There are three equilibrium points $(x^*,y^*)$ in total. They are $(0,0),(0,2)$ and $(3,0)$ respectively. The Jocabian matrix in this case is
\begin{equation*}
    A_{(0,0)} =
    \left[\begin{array}{cc}
    3 & 0 \\
    0 & 2
    \end{array}\right], \; A_{(0,2)} =
    \left[\begin{array}{cc}
    7 & 0 \\
    -2 & -2
    \end{array}\right],   
    \end{equation*}
    \begin{equation*}
    A_{(3,0)} =
    \left[\begin{array}{cc}
    -3 & 6 \\
    0 & -1
    \end{array}\right].
\end{equation*}

The characteristic equation of the system linearized around \mbox{(0,0)} is
$\lambda^2 -5 \lambda + 6 = 0,$ which gives the eigenvalues $\lambda_{1} = 3$ and $\lambda_{2} = 2$. The equilibrium $(0,0)$ is an unstable node. 
The characteristic equation of the system linearized around \mbox{(0,2)} is
$\lambda^2 -5 \lambda - 14 = 0,$ which gives the eigenvalues $\lambda_1 = -2$ and $\lambda_2 = 7$. The equilibrium $(0,2)$ is a saddle node. 
The characteristic equation of the system linearized around \mbox{(3,0)} is
$\lambda^2 +4 \lambda + 3 = 0,$ which gives the eigenvalues $\lambda_1 = -3$ and $\lambda_2 = -1$. The equilibrium $(3,0)$ is a stable point.

In this case, the relationship between the two species is $x$ preying on $y$. This implies that $x$ tends to increase if the relative amount of $y$ is huge and vice versa.   

\subsection{(c,e)=(2,1)} 
%(2,1)
When $(c,e)=(2,1)$, the Jacobian matrix can be written as
\begin{equation*}
    \frac{\partial\textbf{f}}{\partial \textbf{x}}(\textbf{x}) =
    \left[\begin{array}{cc}
    3-2x+2y & 2x \\
    y & 2+x-2y
    \end{array}\right],
\end{equation*}
There are three equilibrium points $(x^*,y^*)$ in total. They are $(0,0),(0,2)$ and $(3,0)$ respectively. The Jocabian matrix in this case is
\begin{equation*}
    A_{(0,0)} =
    \left[\begin{array}{cc}
    3 & 0 \\
    0 & 2
    \end{array}\right], \; A_{(0,2)} =
    \left[\begin{array}{cc}
    7 & 0 \\
    2 & -2
    \end{array}\right],   
    \end{equation*}
    \begin{equation*}
    A_{(3,0)} =
    \left[\begin{array}{cc}
    -3 & 6 \\
    0 & 5
    \end{array}\right].
\end{equation*}

The characteristic equation of the system linearized around \mbox{(0,0)} is
$\lambda^2 -5 \lambda + 6 = 0,$ which gives the eigenvalues $\lambda_{1} = 3$ and $\lambda_{2} = 2$. The equilibrium $(0,0)$ is an unstable node. 
The characteristic equation of the system linearized around \mbox{(0,2)} is
$\lambda^2 -5 \lambda - 14 = 0,$ which gives the eigenvalues $\lambda_1 = -2$ and $\lambda_2 = 7$. The equilibrium $(0,2)$ is a saddle node. 
The characteristic equation of the system linearized around \mbox{(3,0)} is
$\lambda^2 -2 \lambda - 15 = 0,$ which gives the eigenvalues $\lambda_1 = -3$ and $\lambda_2 = 5$. The equilibrium $(3,0)$ is a saddle point.

In this case, the relationship between the two species is symbiotic. This implies that one species would increase if the other one increases. 

\section{Problem 3 - Phase portrait and equilibrium interpretation for one specific case}
\label{sec:prob3} 
Like previous problem, the dynamic behavior of the population model will be analyzed for one specific case. Phase portrait would be obtained by using pplane. And the equilibrium point(s) should be identified analytically. The results should be interpreted under the physical context of two species. 


\section{Solution 3 - Phase portrait and equilibrium interpretation for one specific case}
\label{sec:solu3}
When 
$a=e=1$, $b=f=0$, and $c=d=-1$,
  we can obtain the differential equations as \cref{eqn:PPSystem1,eqn:PPSystem2}\par 
\begin{subequations}\label{eqn:PPSystem}
\begin{align}
    \dot{x} = x(1-y) \label{eqn:PPSystem1} \\
    \dot{y} = y(-1+x) \label{eqn:PPSystem2}
\end{align}
\end{subequations}
The Jacobian matrix is defined as $A = \left. \frac{\partial\textbf{f}}{\partial \textbf{x}}(\textbf{x}) \right|_{\textbf{x}=\textbf{x}_0}$. The Jacobian matrix is calculated to be
\begin{equation*}
    \frac{\partial\textbf{f}}{\partial \textbf{x}}(\textbf{x}) =
    \left[\begin{array}{cc}
    1-y & -x \\
    y & -1+x
    \end{array}\right],
\end{equation*}
By calculating $(x^*,y^*)$, the system has two equilibrium points $(0,0)$ and $(1,1)$.The Jocabian matrix in this case is
\begin{equation*}
    A_{(0,0)} =
    \left[\begin{array}{cc}
    1 & 0 \\
    0 & -1
    \end{array}\right], \; A_{(1,1)} =
    \left[\begin{array}{cc}
    0 & -1 \\
    1 & 0
    \end{array}\right],   
    \end{equation*}
 The characteristic equation of the system linearized around \mbox{(0,0)} is
$\lambda^2 -1 = 0,$ which gives the eigenvalues $\lambda_{1,2} = \pm 1$. The equilibrium $(0,0)$ is a saddle point. 
The characteristic equation of the system linearized around \mbox{(1,1)} is
$\lambda^2 +1 = 0,$ which gives the eigenvalues $\lambda_{1,2} = \pm j$. The equilibrium $(1,1)$ is a non-hyperbolic point so the stability needs to be checked with other methods. From the plot we can infer that this equilibrium is a stable point.    


\section{Problem 5 - Generalize the population model}
\label{sec:prob5} 
So far the popular model is analyzed when the number of the species is two. A more general model where the species $N$ are more than two could be built based on the previous regular. Some elaboration would be included. 

\section{Solution 5 - Generalize the population model}
\label{sec:solu5}
When the number of species $N>2$, a general equation to express the population model can be deduced as~(\ref{eqn:p5}).
\begin{equation}\label{eqn:p5}
   \dot{x}_i = x_i(\alpha_i+ \sum_{j=1}^{N} \beta_{ij}x_j), ~i=1...N
\end{equation}
The coefficients $\beta_{ij}$ will be decided based on the relationship among species further. Intuitively, when $x_j$ preys on $x_i$, then $\beta_{ij}<0, \beta_{ji}>0$;when $x_i$ preys on $x_j$, then $\beta_{ij}>0, \beta_{ji}<0$;when $x_j$ and $x_i$ are competitive, then both $\beta_{ij}$ and $\beta_{ij}$ will be negative;when $x_j$ and $x_i$ are symbiotic, then both $\beta_{ij}$ and $\beta_{ij}$ will be positive;$\beta_{ij}$ and $\beta_{ji}$ might be zero if the relationship between $x_i$ and $x_j$ keeps unknown. 

\begin{figure*}[p] % The * makes the figure span both columns, p places the figure on a float page
  \begin{center}
    \includegraphics[width = 0.9\textwidth, height = 0.5\textwidth]{-2-1}
  \end{center}
  \caption{Phase portrait of the system in \eqref{eqn:PPSystem}, which has an unstable node at \mbox{(0,0)}, two stable points at \mbox{(0,2)} and \mbox{(3,0)} and a saddle point at \mbox{(1,1)}. All the equilibriums are marked by large dots and selected trajectories are marked by solid lines. This figure was generated using PPLANE (\texttt{http://math.rice.edu/~dfield/dfpp.html}).}
  \label{fig:pplane}
\end{figure*}

\begin{figure*}[p] % The * makes the figure span both columns, p places the figure on a float page
  \begin{center}
    \includegraphics[width = 0.9\textwidth, height = 0.5\textwidth]{-21}
  \end{center}
  \caption{Phase portrait of the system in \eqref{eqn:PPSystem}, which has an unstable node at \mbox{(0,0)}, a stable point at \mbox{(0,2)} and a saddle at \mbox{(3,0)}. All the equilibriums are marked by large dots and selected trajectories are marked by solid lines. This figure was generated using PPLANE (\texttt{http://math.rice.edu/~dfield/dfpp.html}).}
  \label{fig:pplane2}
\end{figure*}

\begin{figure*}[p] % The * makes the figure span both columns, p places the figure on a float page
  \begin{center}
    \includegraphics[width = 0.9\textwidth, height = 0.5\textwidth]{2-1}
  \end{center}
  \caption{Phase portrait of the system in \eqref{eqn:PPSystem}, which has an unstable node at \mbox{(0,0)}, a saddle point at \mbox{(0,2)} and a stable at \mbox{(3,0)}. All the equilibriums are marked by large dots and selected trajectories are marked by solid lines. This figure was generated using PPLANE (\texttt{http://math.rice.edu/~dfield/dfpp.html}).}
  \label{fig:pplane2}
\end{figure*}

\begin{figure*}[p] % The * makes the figure span both columns, p places the figure on a float page
  \begin{center}
    \includegraphics[width = 0.9\textwidth, height = 0.5\textwidth]{21}
  \end{center}
  \caption{Phase portrait of the system in \eqref{eqn:PPSystem}, which has an unstable node at \mbox{(0,0)}, two saddle points at \mbox{(0,2)} \mbox{(3,0)}. All the equilibriums are marked by large dots and selected trajectories are marked by solid lines. This figure was generated using PPLANE (\texttt{http://math.rice.edu/~dfield/dfpp.html}).}
  \label{fig:pplane2}
\end{figure*}

\begin{figure*}[p] % The * makes the figure span both columns, p places the figure on a float page
  \begin{center}
    \includegraphics[width = 0.9\textwidth, height = 0.5\textwidth]{p3}
  \end{center}
  \caption{Phase portrait of the system in \eqref{eqn:PPSystem}, which has a saddle node at \mbox{(0,0)} and a stable point at \mbox{(1,1)}. All the equilibriums are marked by large dots and selected trajectories are marked by solid lines. This figure was generated using PPLANE (\texttt{http://math.rice.edu/~dfield/dfpp.html}).}
  \label{fig:pplane2}
\end{figure*}  
\end{document}