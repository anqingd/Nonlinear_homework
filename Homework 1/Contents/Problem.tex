The change in population of two interacting species inhabiting a grassy island can be described by the following equations:
\begin{equation}\label{eqn:Model}
\begin{aligned} 
\dot{x}&=x(t)(a+bx(t)+cy(t)) \\
\dot{y}&=y(t)(d+ex(t)+fy(t))
\end{aligned}
\end{equation}
where $a, b, ... ,f$ are parameters and $x, y$ are non-negative continuous function with respect to time. The model can be effective only with two preconditions fulfilled. First, the populations here are normalized and therefore can be thought of as continuous numbers. Second, the habitat is isolated area, which means the impact from other creatures can be neglected. While $a, b, d$ and $f$ are very much dependent on food supply and intraspecific competition, $c$ and $e$ reflect biological interactions between those two species. This work seeks to provide more details of this model by analyzing the solutions of \eqref{eqn:Model} and their biological backgrounds when those parameters are assigned with different values. A generalized model for more species is presented in the end.